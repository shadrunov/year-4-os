\section{Цель работы}
Целью работы является изучение основных политик, их подкатегорий и процесса настройки
расширенной политики аудита.

\section{Ход работы}
\subsection{Рекомендуемые правила}

Для системы auditd существуют рекомендуемые правила (\url{https://github.com/Neo23x0/auditd/blob/master/audit.rules}). Эти правила содержат настройки для аудита следующих действий:
\begin{itemize}
    \item Доступ к логам auditd (успех и неуспех)
    \item Доступ к правилам auditd (файлы конфигурации и утилиты)
    \item Доступ к параметрам ядра sysctl
    \item Доступ к модулям ядра: insmod, modprobe, rmmod, файлы конфигурации
    \item Монтирование mount
    \item Файл подкачки: swapon, swapoff
    \item Служба времени: adjtimex, settimeofday, clock\_settime, /etc/localtime
    \item Stunnel
    \item Cron: /etc/crontab
    \item Пользователи, группы, пароли: /etc/group, /etc/passwd, /etc/shadow
    \item Sudo: /etc/sudoers
    \item Passwd: /usr/bin/passwd
    \item Изменение групп: groupadd, groupmod, addgroup, useradd, userdel, usermod, adduser
    \item Сеть: sethostname, setdomainname
    \item Удалённая консоль: /bin/bash connect
    \item Успешные ipv4 и ipv6 подключения
    \item Файлы: /etc/hosts, /etc/sysconfig/network, /etc/network
    \item Init
    \item Пути к подключаемым библиотекам: /etc/ld.so.conf, /etc/ld.so.preload
    \item Настройки Pam, Mail
    \item SSH: /etc/ssh/sshd\_config, /root/.ssh
    \item Systemd: /bin/systemctl, /etc/systemd/
    \item SELinux
    \item Ошибки доступа к каталогам в корне файловой системы
    \item Повышение привилегий: su, sudo
    \item Питание: shutdown, poweroff, reboot, halt
    \item Сессии: utmp, btmp, wtmp
    \item Дискреционный доступ: chmod, chown etc.
    \item Правила на разведку, на подозрительную активность, работу с архивами
    \item Нестандартные командные оболочки, профили оболочен 
    \item Инъекция кода с ptrace
    \item Анонимное создание файлов
    \item Доступ администраторов в чужие рабочие директории 
    \item Создание сокетов
    \item Использование пакетных менеджеров
    \item Grep
    \item Docker и виртуализация (qemu, virtual box)
    \item kubelet
    \item Запуск команд под рутом
    \item Неуспех доступа к файлам
\end{itemize}

Установим правила (рисунки 1-3):
\image{11.png}{Сохранение правил в файл}{1}
\FloatBarrier

\image{12.png}{Применение правил}{1}
\FloatBarrier

\image{13.png}{Количество активных правил}{1}
\FloatBarrier

Рассмотрим основные действия, которые можно логировать, и изучим, как рекомендуемые правила справляются в рассматриваемых случаях.

\subsection{События входа и выхода}
Для логирования событий входа и выхода подойдёт следующий участок правил:

\begin{lstlisting}[language=bash, numbers=none, caption={События входа и выхода}]
## Session initiation information
-w /var/run/utmp -p wa -k session
-w /var/log/btmp -p wa -k session
-w /var/log/wtmp -p wa -k session
\end{lstlisting}

Файл utmp содержит данные о текущей сессии, btmp — об ошибках, wtmp — история сессий. Сделаем поиск событий по ключу session. Видно, что последний вход под пользователем test осуществлялся в 22:20 (рисунок 4). В это время есть логи аудита (рисунок 5):

\image{21.png}{Последний вход}{0.7}
\FloatBarrier

\image{22.png}{Логи аудита}{1}
\FloatBarrier


\subsection{Управление учётными записями}
Для логирования административных действий подходит следующий раздел:

\begin{lstlisting}[language=bash, numbers=none, caption={Управление учётными записями}]
## User, group, password databases
-w /etc/group -p wa -k etcgroup
-w /etc/passwd -p wa -k etcpasswd
-w /etc/gshadow -k etcgroup
-w /etc/shadow -k etcpasswd
-w /etc/security/opasswd -k opasswd

## Tools to change group identifiers
-w /usr/sbin/groupadd -p x -k group_modification
-w /usr/sbin/groupmod -p x -k group_modification
-w /usr/sbin/addgroup -p x -k group_modification
-w /usr/sbin/useradd -p x -k user_modification
-w /usr/sbin/userdel -p x -k user_modification
-w /usr/sbin/usermod -p x -k user_modification
-w /usr/sbin/adduser -p x -k user_modification
\end{lstlisting}

Проверим создание и удаление пользователя по ключу user\_modification. Создадим и удалим пользователя (рисунок 6). Соответствующие записи появились в логах аудита (рисунок 7).

\image{31.png}{Создание и удаление пользователя}{1}
\FloatBarrier

\image{32.png}{Логи аудита}{1}
\FloatBarrier


\subsection{Аудит изменения политики}
Для логирования изменений в политику auditd подходит следующий раздел:

\begin{lstlisting}[language=bash, numbers=none, caption={Аудит изменения политики}]
# Audit the audit logs
### Successful and unsuccessful attempts to read information from the audit records
-w /var/log/audit/ -p wra -k auditlog
-w /var/audit/ -p wra -k auditlog

## Auditd configuration
### Modifications to audit configuration that occur while the audit collection functions are operating
-w /etc/audit/ -p wa -k auditconfig
-w /etc/libaudit.conf -p wa -k auditconfig
-w /etc/audisp/ -p wa -k audispconfig

## Monitor for use of audit management tools
-w /sbin/auditctl -p x -k audittools
-w /sbin/auditd -p x -k audittools
-w /usr/sbin/auditd -p x -k audittools
-w /usr/sbin/augenrules -p x -k audittools

\end{lstlisting}

Выполним команду auditctl и проверим, появились ли логи (рисунки 8, 9). 

\image{41.png}{auditctl}{1}
\FloatBarrier

\image{42.png}{Логи аудита}{1}
\FloatBarrier



\subsection{Аудит использования привилегий}
Для логирования использования привилегий подходит следующий раздел:

\begin{lstlisting}[language=bash, numbers=none, caption={Аудит использования привилегий}]
## Process ID change (switching accounts) applications
-w /bin/su -p x -k priv_esc
-w /usr/bin/sudo -p x -k priv_esc
\end{lstlisting}

Выполним команду sudo -i и проверим, появились ли логи (рисунки 10, 11). 

\image{51.png}{sudo -i}{1}
\FloatBarrier

\image{52.png}{Логи аудита}{1}
\FloatBarrier


\subsection{Аудит доступа пользователя к объектам}
Для логирования неудачных попыток пользователей получить доступ к разделам файловой системы подходит следующий фрагмент правил:

\begin{lstlisting}[language=bash, numbers=none, caption={Аудит доступа пользователя к объектам}]

## File Access
### Unauthorized Access (unsuccessful)
-a always,exit -F arch=b64 -S creat -S open -S openat -S open_by_handle_at -S truncate -S ftruncate -F exit=-EACCES -F auid>=1000 -F auid!=-1 -k file_access
-a always,exit -F arch=b64 -S creat -S open -S openat -S open_by_handle_at -S truncate -S ftruncate -F exit=-EPERM -F auid>=1000 -F auid!=-1 -k file_access

### Unsuccessful Creation
-a always,exit -F arch=b64 -S mkdir,creat,link,symlink,mknod,mknodat,linkat,symlinkat -F exit=-EACCES -k file_creation
-a always,exit -F arch=b64 -S mkdir,link,symlink,mkdirat -F exit=-EPERM -k file_creation

### Unsuccessful Modification
-a always,exit -F arch=b64 -S rename -S renameat -S truncate -S chmod -S setxattr -S lsetxattr -S removexattr -S lremovexattr -F exit=-EACCES -k file_modification
-a always,exit -F arch=b64 -S rename -S renameat -S truncate -S chmod -S setxattr -S lsetxattr -S removexattr -S lremovexattr -F exit=-EPERM -k file_modification
\end{lstlisting}

Попробуем изменить файл /etc/shadow и проверим, появились ли логи об отказе в доступе (рисунки 10, 11). 

\image{61.png}{truncate}{1}
\FloatBarrier

\image{62.png}{Логи аудита}{1}
\FloatBarrier



\section{Выводы о проделанной работе}
В ходе работы рассмотрены возможности, варианты применения и настройки системы audit, а также ее изначальные параметры. Получены знания о механизмах обеспечения безопасности ОС Linux, навыки настроек правил auditd, протоколирование важной с точки зрения безопасности информации.
